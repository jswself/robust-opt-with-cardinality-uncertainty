\documentclass[letterpaper, 10pt]{article}

\usepackage[margin=1.0in]{geometry}

\usepackage{amsmath} \allowdisplaybreaks
\usepackage{amssymb}
\usepackage{amsthm} \theoremstyle{definition}
\newtheorem{theorem}{Theorem}
\newtheorem{lemma}{Lemma}
\newtheorem{assumption}{Assumption}
\newtheorem{proposition}{Proposition}
\newtheorem{definition}{Definition}


\usepackage{booktabs}
\usepackage{graphicx}
\usepackage{appendix}
\usepackage{hyperref}
\usepackage{url}

\newcommand{\email}[1]{{\href{mailto:#1}{\nolinkurl{#1}}}}


\usepackage{xcolor}
\newcommand{\note}[1]{{\Large\bf#1}}
\newcommand{\bluenote}[1]{{\Large\color{blue}#1}}
\newcommand{\rednote}[1]{{\Large\color{red}#1}}

\usepackage{etoolbox}
% To use, for example, \Ac for \mathcal{A}.  Requires "etoolbox" package
\makeatletter
\def\do#1{\@namedef{#1c}{\ensuremath{\mathcal{#1}}}}
\docsvlist{A,B,C,D,E,F,G,H,I,J,K,L,M,N,O,P,Q,R,S,T,U,V,W,X,Y,Z}
\makeatother

\def\Eb{\mathbb{E}}
\def\Rb{\mathbb{R}}


\usepackage{authblk}
\renewcommand\Authfont{\sf\Large}
\renewcommand\Affilfont{\rm\small}

\title{A Strategy for solving the Robust Combinatorial Optimization Problems with Cardinality Constrained Uncertainty}
\author{
    Sangwoo Cho\footnote{Corresponding Author: \email{jswself@kaist.ac.kr}}\\Department of Industrial and SystemsEngineering\\KAIST, Daejeon, Korea\\
    \and
    Changhyun Kwon\footnote{Corresponding Author: \email{chkwon@kaist.ac.kr}}\\Department of Industrial and Systems Engineering\\KAIST, Daejeon, Korea
    }
\date{September 23, 2024}

%\usepackage{showlabels} 
%\usepackage[final]{showlabels}



\usepackage[]{natbib}

\usepackage{enumerate}
\def\Ac{\mathcal{A}}
\def\Nc{\mathcal{N}}
\def\Lc{\mathcal{L}}

\usepackage{microtype}
\usepackage{mathtools}
\DeclarePairedDelimiter{\ceil}{\lceil}{\rceil}
\DeclarePairedDelimiter{\floor}{\lfloor}{\rfloor}

\usepackage{bm}
\usepackage{booktabs}
\usepackage{color}
\newcommand{\comment}[1]{ {\Large\color{red}#1} }
\renewcommand{\vec}[1]{\bm{#1}}




\begin{document}
\maketitle

\begin{abstract}
Robust combinatorial optimization problems with cardinality constrained uncertainty may be solved by a finite number of nominal problems. In this paper, we show that the number of nominal problems to be solved can be reduced significantly. \\[0.5em]
\noindent\textbf{Keywords:} robust combinatorial optimization; discrete optimization
\end{abstract}







\section{Robust combinatorial optimization problem}
\label{sec:2}

Let $X \subset \{0, 1\}^n$ be a set of feasible solutions of a combinatorial optimization problem. The nominal combinatorial optimization problem of our interest is defined as follows:
\begin{equation}
\min_{\vec{x}\in X} \vec{c}^\top \vec{x}
\end{equation}
where $\vec{c}^\top \vec{x} = \sum_{j=1}^n c_j x_j$. \citet{Bertsimas2003} considered uncertainty for objective coefficients such that the cost of item $j \in N =\{1,2,\cdots,n\}$ takes a value in the interval $[c_j, c_j + d_j]$, where $d_j \geq 0$. A robust combinatorial optimization problem is considered in the following form:
\begin{equation} \label{problem}
Z^* = \min_{\vec{x}\in X} \bigg\{\vec{c}^\top \vec{x} + \max_{\{S|S \subset N, |S| \leq \Gamma\}} \sum_{j \in S} d_j x_j \bigg\}
\end{equation}
where at most $\Gamma$ components of the cost coefficients can be $c_j+d_j$; hence, the uncertainty set is called cardinality constrained. The budget of uncertainty $\Gamma$ is a positive integer and represents the risk attitude of decision makers, and $1 \leq \Gamma \leq n$. Without loss of generality, we assume that the indices are sorted in descending order of the size of $d_i$ and define $d_{n+1}=0$ so that 
\begin{equation}
	d_1 \geq d_2 \geq \cdots \geq d_n \geq d_{n+1} = 0.
\end{equation}

\citet{Bertsimas2003} showed that \eqref{problem} is equivalent to
\begin{equation} \label{bs_dual}
	Z^* = \min_{\vec{x}\in X, \theta\geq0} \bigg(
		\Gamma \theta + \vec{c}^\top \vec{x} + \sum_{j\in N} \max(d_j - \theta, 0) x_j 
	\bigg)
\end{equation}
and it can be solved by solving $n+1$ nominal problems. In particular,
\begin{equation} \label{main}
Z^* = \min_{l=1,2,\cdots,n+1} G^l,
\end{equation}
where for $ l = 1, 2, \dots, n+1$:
\begin{equation} \label{Gmain}
G^l = \Gamma d_l + \min_{\vec{x} \in X} \bigg( \vec{c}^\top \vec{x} + \sum_{j=1}^l (d_j - d_l) x_j \bigg)
\end{equation}
We let $x_{n+1}=0$, so that \eqref{Gmain} is well-defined. 

This result is very useful, because we can solve the robust optimization problem by solving a finite number of nominal problems. 
If the nominal problem can be solved in polynomial time, we can also solve the corresponding robust problem in polynomial time. 
\citet{Park2007} showed that the number of nominal problems to be solved can be reduced to $n-\Gamma+1$, and \citet{Alvarez2013} to $n-\Gamma+2$ independently.
From the previous paper, they showed that the number of nominal problems to be solved can be further reduced to $\lceil \frac{n-\Gamma}{2} \rceil + 1$.
Here, we show that the number of nomial problems can be cut down to $\lceil \frac{n-\Gamma}{4} \rceil + 1$ by the heuristic.

\section{Previous Results}

For a feasible solution $\vec{x} \in X$, let 
\begin{equation}
G^l (\vec{x}) = \Gamma d_l + \vec{c}^\top \vec{x} + \sum_{j=1}^l (d_j - d_l) x_j \qquad \forall l=1,2,...,n+1
\end{equation}
then $G^l$ in \eqref{Gmain} can be written as
\begin{equation}
G^l = \min_{\vec{x} \in X} G^l (\vec{x})
\end{equation}
We let $G^0(\vec{x})=G^1(\vec{x})$ for notational simplicity. We also define $\vec{x}^l\in X$ such that
\begin{equation} \label{xldef}
G^l = G^l(\vec{x}^l) \leq G^l(\vec{x}) \quad \forall \vec{x}\in X
\end{equation}


We first consider $G^{l+1}(\vec{x}) - G^l(\vec{x})$ for $l=1,...,n$ and $\vec{x}\in X$:
\begin{align}
G^{l+1}(\vec{x}) - G^l(\vec{x}) 
&= \Gamma d_{l+1} + \sum_{j=1}^n c_j x_j + \sum_{j=1}^{l+1} (d_j-d_{l+1})x_j \nonumber\\
&\qquad -  \Gamma d_{l} - \sum_{j=1}^n c_j x_j - \sum_{j=1}^{l} (d_j-d_{l})x_j \nonumber\\
&= \Gamma (d_{l+1}-d_l) + \sum_{j=1}^l (d_l - d_{l+1}) x_j \nonumber\\
&= (d_{l+1} - d_l) \bigg( \Gamma - \sum_{j=1}^l x_j \bigg ) \label{diff1}
\end{align}
Similarly, we consider for $l=2,...,n+1$ and $\vec{x}\in X$:
\begin{equation} \label{diff2}
G^{l}(\vec{x}) - G^{l-1}(\vec{x}) = (d_l - d_{l-1}) \bigg( \Gamma - \sum_{j=1}^{l-1} x_j \bigg )
\end{equation}

Using \eqref{diff1} and \eqref{diff2}, the following lemmas were provided.

\begin{lemma} \label{lem:prop}
For $l=1,2,...,n$ and for any $\vec{x}\in X$, the following holds:
\begin{enumerate}

\item If $\sum_{j=1}^l x_j \leq \Gamma$, then $G^{l-1}(\vec{x}) \geq G^l(\vec{x}) \geq G^{l+1}(\vec{x})$.

\item If $\sum_{j=1}^l x_j > \Gamma$, then $G^{l-1}(\vec{x}) \leq G^l(\vec{x}) \leq G^{l+1}(\vec{x})$.

\end{enumerate}
\end{lemma}

\begin{lemma} \label{lem:prop3}
For any $\vec{x}\in X$, we have 
$
	G^1(\vec{x}) \geq G^2(\vec{x}) \geq \cdots \geq G^\Gamma(\vec{x}) \geq G^{\Gamma+1} (\vec{x})
$.
Furthermore, 
$
G^1 \geq G^2 \geq ... \geq G^\Gamma \geq G^{\Gamma+1}
$.
\end{lemma}

\begin{lemma} \label{lem:prop5}
For any $l=1,2,...,n$, we have either $G^l\geq G^{l+1}$ or $G^l\geq G^{l-1}$.

\end{lemma}

\section{New Result}
The proposition expands the conclusion of Lemma \ref{lem:prop5} as a slightly reduced form.

\begin{lemma} \label{lem:new_prop}
	Let $x^{l}$ be an optimal solution of $G^{l}$. For any $l \in \{2, \ldots, n-1 \}$,
	\begin{enumerate}
		\item If $\sum_{j=1}^{l}{x_{j}} \le \Gamma - 1$ or $\sum_{j=1}^{l-2}{x_{j}} \ge \Gamma$, then $G^{l} \ge G^{l+2}$ or $G^{l} \ge G^{l-2}$.
		\item Otherwise, $G^{l} \le G^{l+2}$ and $G^{l} \le G^{l-2}$.
	\end{enumerate}
\end{lemma}
\begin{proof}
	1. We first consider $G^{l+2}(x) - G^{l}(x)$ and then, $G^{l-2}(x) - G^{l}(x)$. We have \rednote{Need to use vector notation $\vec{x}$...}
	\begin{align*}
		G^{l+2}(x) - G^{l}(x) 
		& = \{ G^{l+2}(x) - G^{l+1}(x) \} + \{ G^{l+1}(x) - G^{l}(x) \} \\
		& = (d_{l+2} - d_{l}) \left( \Gamma - \sum\limits_{j=1}^{l}{x_{j}} \right) - x_{l+1}(d_{l+2} - d_{l+1}).
	\end{align*}
	Similarly,
	\[
		G^{l-2}(x) - G^{l}(x) = -(d_{l} - d_{l-2}) \left( \Gamma - \sum\limits_{j=1}^{l-2}{x_{j}} \right) + x_{l-1}(d_{l} - d_{l-1}).
	\]
	Suppose that $\sum_{j=1}^{l}{x_{j}} \le \Gamma - 1$. We obtain
	\begin{align*}
		G^{l+2}(x) - G^{l}(x) 
		&= (d_{l+2} - d_{l}) \left( \Gamma - \sum\limits_{j=1}^{l}{x_{j}} \right) - x_{l+1}(d_{l+2} - d_{l+1}) \\
		&\le d_{l+2} - d_{l} + d_{l+1} - d_{l+2} \\
		&= -(d_{l} - d_{l+1}) \\
		&\le 0.
	\end{align*}
	Thus,
	\[
		G^{l+2} = G^{l+2}(\vec{x}^{l+2}) \le G^{l+2}(\vec{x}^{l}) \le G^{l}(\vec{x}^{l}) = G^{l}.
	\]
	Also, suppose that $\sum_{j=1}^{l-2}{x_{j}} \ge \Gamma$. It gives
	\begin{align*}
		G^{l-2}(x) - G^{l}(x) 
		&= -(d_{l} - d_{l-2}) \left( \Gamma - \sum\limits_{j=1}^{l-2}{x_{j}} \right) + x_{l-1}(d_{l} - d_{l-1}) \\
		&\le 0+0 \le 0.
	\end{align*}
	Thus,
	\[
		G^{l-2} = G^{l-2}(\vec{x}^{l-2}) \le G^{l-2}(\vec{x}^{l}) \le G^{l}(\vec{x}^{l}) = G^{l}.
	\]
	These two statements complete the first part of the proof. 

	\rednote{
		From the proof written above, the correct statement is:
		
		\begin{enumerate}
		\item If $\sum_{j=1}^{l}{x_{j}} \le \Gamma - 1$, then $G^{l} \ge G^{l+2}$.
		\item If $\sum_{j=1}^{l-2}{x_{j}} \ge \Gamma$, then $G^{l} \ge G^{l-2}$.
		\item Otherwise.... 
		\end{enumerate}
	}


	2.
	We assume that $\sum\limits_{j=1}^{l}{x_{j}} \ge \Gamma$ and $\sum\limits_{j=1}^{l-2}{x_{j}} \le \Gamma-1$.
	Note that $\vec{x}^{l}$ is a binary vector. There are only 3 cases that satisfy the assumption;
	\[
		\left( \sum\limits_{j=1}^{l-2}{x_{j}}, \sum\limits_{j=1}^{l}{x_{j}}  \right) \in \bigg\{ (\Gamma - 2, \Gamma), (\Gamma - 1, \Gamma), (\Gamma - 1, \Gamma + 1) \bigg\}
	\]
	since the difference between the two series is at most 2.
	\begin{enumerate}
		\item If $\left( \sum\limits_{j=1}^{l-2}{x_{j}}, \sum\limits_{j=1}^{l}{x_{j}}  \right) = (\Gamma - 2, \Gamma)$,
		\begin{enumerate}
			\item $G^{l+2}(\vec{x}) - G^{l}(\vec{x}) = (d_{l+2} - d_{l}) \left( \Gamma - \sum\limits_{j=1}^{l}{x_{j}} \right) - x_{l+1}(d_{l+2} - d_{l+1}) = x_{l+1}(d_{l+1} - d_{l+2}) \ge 0$. Then,
			\begin{center}
				$G^{l} = G^{l}(\vec{x}^{l}) \le G^{l}(\vec{x}^{l+2}) \le G^{l+2}(\vec{x}^{l+2}) = G^{l+2}$.
			\end{center}
			\item $G^{l-2}(\vec{x}) - G^{l}(\vec{x}) = -(d_{l} - d_{l-2}) \left( \Gamma - \sum\limits_{j=1}^{l-2}{x_{j}} \right) + x_{l-1}(d_{l} - d_{l-1}) = 2d_{l-2} - 2d_{l} + x_{l-1}(d_{l} - d_{l-1}) \ge 2d_{l-2} - 2d_{l} + d_{l} - d_{l-1} = (d_{l-2} - d_{l-1}) + (d_{l-2} - d_{l}) \ge 0$. Then,
			\begin{center}
				$G^{l} = G^{l}(\vec{x}^{l}) \le G^{l}(\vec{x}^{l-2}) \le G^{l-2}(\vec{x}^{l-2}) = G^{l-2}$.
			\end{center}
		\end{enumerate}
		\item If $\left( \sum\limits_{j=1}^{l-2}{x_{j}}, \sum\limits_{j=1}^{l}{x_{j}}  \right) = (\Gamma - 1, \Gamma)$,
		\begin{enumerate}
			\item $G^{l+2}(\vec{x}) - G^{l}(\vec{x}) = (d_{l+2} - d_{l}) \left( \Gamma - \sum\limits_{j=1}^{l}{x_{j}} \right) - x_{l+1}(d_{l+2} - d_{l+1}) = x_{l+1}(d_{l+1} - d_{l+2}) \ge 0$. Then,
			\begin{center}
				$G^{l} = G^{l}(\vec{x}^{l}) \le G^{l}(\vec{x}^{l+2}) \le G^{l+2}(\vec{x}^{l+2}) = G^{l+2}$.
			\end{center}
			\item $G^{l-2}(\vec{x}) - G^{l}(\vec{x}) = -(d_{l} - d_{l-2}) \left( \Gamma - \sum\limits_{j=1}^{l-2}{x_{j}} \right) + x_{l-1}(d_{l} - d_{l-1}) =  d_{l-2} - d_{l} + x_{l-1}(d_{l} - d_{l-1}) \ge d_{l-2} - d_{l} + d_{l} - d_{l-1} = (d_{l-2} - d_{l-1}) \ge 0$. Then,
			\begin{center}
				$G^{l} = G^{l}(\vec{x}^{l}) \le G^{l}(\vec{x}^{l-2}) \le G^{l-2}(\vec{x}^{l-2}) = G^{l-2}$.
			\end{center}
		\end{enumerate}
		\item If $\left( \sum\limits_{j=1}^{l-2}{x_{j}}, \sum\limits_{j=1}^{l}{x_{j}}  \right) = (\Gamma - 1, \Gamma+1)$,
		\begin{enumerate}
			\item $G^{l+2}(\vec{x}) - G^{l}(\vec{x}) = (d_{l+2} - d_{l}) \left( \Gamma - \sum\limits_{j=1}^{l}{x_{j}} \right) - x_{l+1}(d_{l+2} - d_{l+1}) = (d_{l} - d_{l+2}) + x_{l+1}(d_{l+1} - d_{l+2}) \ge 0$. Then,
			\begin{center}
				$G^{l} = G^{l}(\vec{x}^{l}) \le G^{l}(\vec{x}^{l+2}) \le G^{l+2}(\vec{x}^{l+2}) = G^{l+2}$.
			\end{center}
			\item $G^{l-2}(\vec{x}) - G^{l}(\vec{x}) = -(d_{l} - d_{l-2}) \left( \Gamma - \sum\limits_{j=1}^{l-2}{x_{j}} \right) + x_{l-1}(d_{l} - d_{l-1}) =  d_{l-2} - d_{l} + x_{l-1}(d_{l} - d_{l-1}) \ge d_{l-2} - d_{l} + d_{l} - d_{l-1} = (d_{l-2} - d_{l-1}) \ge 0$. Then,
			\begin{center}
				$G^{l} = G^{l}(\vec{x}^{l}) \le G^{l}(\vec{x}^{l-2}) \le G^{l-2}(\vec{x}^{l-2}) = G^{l-2}$.
			\end{center}
		\end{enumerate}
	\end{enumerate}
	These cases end the rest of the proof.
\end{proof}

\section{New Strategies on searching the Optimal Solution}
Firstly, we may ignore some partial problems by the status of some problems by lemma \ref{lem:new_prop}.

\textbf{Algorithm to find the optimal solution}
\begin{enumerate}
	\item $r=1, L = \{ G^{\Gamma + 1}, G^{\Gamma + 3}, G^{\Gamma + 5}, \ldots, G^{\Gamma + \gamma}, G^{n+1} \}$ where $\gamma = \max\{{2n+1 : \Gamma + 2n + 1 < n+1, n \in \mathbb{N}} \}$.
	\item Find $\vec{x}^{\Gamma + 4r - 1}$, the optimal solution of $G^{\Gamma + 4r - 1}$.
	\item If $\vec{x}^{\Gamma + 4r - 1}$ satisfies lemma \ref{lem:new_prop}-2, remove $4r-3$ and $4r+1$ from $L$.
	\item Erase $G^{\Gamma + 4r - 1}$ from $L$, increase $r$ by 1 and repeat it from 2 until $\Gamma + 4r-1 < n+1$.
	\item Evaluate the rest partial problems in $L$.
\end{enumerate}
We may intuitively ignore the cases which satisfy the lemma \ref{lem:new_prop}, because there are only 3 cases among numerous cases.

\textbf{Algorithm to find the optimal solution}
\begin{enumerate}
	\item $L = \{ G^{\Gamma + 1}, G^{\Gamma + 3}, G^{\Gamma + 5}, \ldots, G^{\Gamma + \gamma}, G^{n+1} \}$ where $\gamma = \max\{{2n+1 : \Gamma + 2n + 1 < n+1, n \in \mathbb{N}} \}$.
	\item Remove $G^{\Gamma + 4k - 3}$ from $L$ for every possible $k \in \mathbb{N}$.
	\item Evaluate the problems in $L$.
\end{enumerate}
We tried to find the case which satisfies the second part of lemma \ref{lem:new_prop}, but there was no such case, whatever $n$ or $\Gamma$ are.
It may conclude that we can find the optimal solution with this heuristic, with a very high chance.
Also, this result implies that the algorithm using lemma \ref{lem:new_prop} will work as similar as the traditional one,
since there is no case taht the third step of algorithm removes the neighbors.

\section{Concluding Remarks}
In this proposition, we could find the heuristic which search the optimal solution before long than the previous alogrithm.
The number of searches the algorithm performs becomes the half, $\lceil \frac{n - \Gamma}{4} \rceil + 1$.





%\bibliographystyle{plainnat}
\bibliographystyle{apalike}
\bibliography{rsp}





\end{document}







%

